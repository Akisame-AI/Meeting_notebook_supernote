\documentclass{article}
\usepackage{tikz}
\usepackage{hyperref}
\usepackage{fontawesome} % Include FontAwesome icons
\usetikzlibrary{calc}
\usepackage[paperheight=158mm,paperwidth=119mm,left=10mm,right=10mm,top=15mm, bottom=10mm,head=11.0pt]{geometry}%defines the shape of your booklet. an extra 3 mm has been added on all sides for the use of chapter tabs.
\pagenumbering{gobble}%suppress numbering

\begin{document}
\noindent
\hypertarget{todolist}{}
\begin{tabular}
{p{0.1\textwidth}|p{0.7\textwidth}|p{0.15\textwidth}}
\hline
\textbf{Origin} & \textbf{To-Do Item} & \textbf{Date} \\ 
\hline
 & & \\[10mm] % Adjust the height as needed
\hline
 & & \\[10mm] 
\hline
 & & \\[10mm] 
\hline
 & & \\[10mm] 
\hline
 & & \\[10mm] 
\hline
 & & \\[10mm] 
\hline
 & & \\[10mm] 
\hline
 & & \\[10mm] 
\hline
 & & \\[10mm] 
\hline
\end{tabular}
\newpage
\noindent
\begin{tabular}
{p{0.1\textwidth}|p{0.7\textwidth}|p{0.15\textwidth}}
\hline
\textbf{Origin} & \textbf{To-Do Item} & \textbf{Date} \\ 
\hline
 & & \\[10mm] % Adjust the height as needed
\hline
 & & \\[10mm] 
\hline
 & & \\[10mm] 
\hline
 & & \\[10mm] 
\hline
 & & \\[10mm] 
\hline
 & & \\[10mm] 
\hline
 & & \\[10mm] 
\hline
 & & \\[10mm] 
\hline
 & & \\[10mm] 
\hline
\end{tabular}
\newpage
\noindent
\begin{tabular}
{p{0.1\textwidth}|p{0.7\textwidth}|p{0.15\textwidth}}
\hline
\textbf{Origin} & \textbf{To-Do Item} & \textbf{Date} \\ 
\hline
 & & \\[10mm] % Adjust the height as needed
\hline
 & & \\[10mm] 
\hline
 & & \\[10mm] 
\hline
 & & \\[10mm] 
\hline
 & & \\[10mm] 
\hline
 & & \\[10mm] 
\hline
 & & \\[10mm] 
\hline
 & & \\[10mm] 
\hline
 & & \\[10mm] 
\hline
\end{tabular}
\newpage
% Index Page with Table
\section*{Index}
\addcontentsline{toc}{section}{Index}
\hypertarget{index1}{}
\begin{tabular}{|p{0.05\textwidth}|p{0.1\textwidth}|p{0.6\textwidth}|p{0.05\textwidth}|p{0.05\textwidth}|}
\hline
\# & Date & Topic & T & P1 \\
% Fill in the table rows within the loop
\hline
   \large{1} & & & \hyperlink{blank1}{\large{\faicon{group}}} & \hyperlink{lines11}{\huge{\faBook}} \\
\hline
   \large{2} & & & \hyperlink{blank2}{\large{\faicon{group}}} & \hyperlink{lines12}{\huge{\faBook}} \\
\hline
   \large{3} & & & \hyperlink{blank3}{\large{\faicon{group}}} & \hyperlink{lines13}{\huge{\faBook}} \\
\hline
   \large{4} & & & \hyperlink{blank4}{\large{\faicon{group}}} & \hyperlink{lines14}{\huge{\faBook}} \\
\hline
   \large{5} & & & \hyperlink{blank5}{\large{\faicon{group}}} & \hyperlink{lines15}{\huge{\faBook}} \\
\hline
   \large{6} & & & \hyperlink{blank6}{\large{\faicon{group}}} & \hyperlink{lines16}{\huge{\faBook}} \\
\hline
   \large{7} & & & \hyperlink{blank7}{\large{\faicon{group}}} & \hyperlink{lines17}{\huge{\faBook}} \\
\hline
   \large{8} & & & \hyperlink{blank8}{\large{\faicon{group}}} & \hyperlink{lines18}{\huge{\faBook}} \\
\hline
   \large{9} & & & \hyperlink{blank9}{\large{\faicon{group}}} & \hyperlink{lines19}{\huge{\faBook}} \\
\hline
   \large{10} & & & \hyperlink{blank10}{\large{\faicon{group}}} & \hyperlink{lines110}{\huge{\faBook}} \\
\hline
   \large{11} & & & \hyperlink{blank11}{\large{\faicon{group}}} & \hyperlink{lines111}{\huge{\faBook}} \\
\hline
   \large{12} & & & \hyperlink{blank12}{\large{\faicon{group}}} & \hyperlink{lines112}{\huge{\faBook}} \\
\hline
   \large{13} & & & \hyperlink{blank13}{\large{\faicon{group}}} & \hyperlink{lines113}{\huge{\faBook}} \\
\hline
   \large{14} & & & \hyperlink{blank14}{\large{\faicon{group}}} & \hyperlink{lines114}{\huge{\faBook}} \\
\hline
   \large{15} & & & \hyperlink{blank15}{\large{\faicon{group}}} & \hyperlink{lines115}{\huge{\faBook}} \\
\hline
   \large{16} & & & \hyperlink{blank16}{\large{\faicon{group}}} & \hyperlink{lines116}{\huge{\faBook}} \\
\hline
   \large{17} & & & \hyperlink{blank17}{\large{\faicon{group}}} & \hyperlink{lines117}{\huge{\faBook}} \\
\hline
   \large{18} & & & \hyperlink{blank18}{\large{\faicon{group}}} & \hyperlink{lines118}{\huge{\faBook}} \\
\hline

\end{tabular}

\newpage
\hypertarget{index2}{}
\noindent
\begin{tabular}{|p{0.05\textwidth}|p{0.1\textwidth}|p{0.6\textwidth}|p{0.05\textwidth}|p{0.05\textwidth}|}
\hline
\# & Date & Topic & T & P1 \\
% Fill in the table rows within the loop
\hline
   \large{19} & & & \hyperlink{blank19}{\large{\faicon{group}}} & \hyperlink{lines119}{\huge{\faBook}} \\
\hline
   \large{20} & & & \hyperlink{blank20}{\large{\faicon{group}}} & \hyperlink{lines120}{\huge{\faBook}} \\
\hline
   \large{21} & & & \hyperlink{blank21}{\large{\faicon{group}}} & \hyperlink{lines121}{\huge{\faBook}} \\
\hline
   \large{22} & & & \hyperlink{blank22}{\large{\faicon{group}}} & \hyperlink{lines122}{\huge{\faBook}} \\
\hline
   \large{23} & & & \hyperlink{blank23}{\large{\faicon{group}}} & \hyperlink{lines123}{\huge{\faBook}} \\
\hline
   \large{24} & & & \hyperlink{blank24}{\large{\faicon{group}}} & \hyperlink{lines124}{\huge{\faBook}} \\
\hline
   \large{25} & & & \hyperlink{blank25}{\large{\faicon{group}}} & \hyperlink{lines125}{\huge{\faBook}} \\
\hline
   \large{26} & & & \hyperlink{blank26}{\large{\faicon{group}}} & \hyperlink{lines126}{\huge{\faBook}} \\
\hline
   \large{27} & & & \hyperlink{blank27}{\large{\faicon{group}}} & \hyperlink{lines127}{\huge{\faBook}} \\
\hline
   \large{28} & & & \hyperlink{blank28}{\large{\faicon{group}}} & \hyperlink{lines128}{\huge{\faBook}} \\
\hline
   \large{29} & & & \hyperlink{blank29}{\large{\faicon{group}}} & \hyperlink{lines129}{\huge{\faBook}} \\
\hline
   \large{30} & & & \hyperlink{blank30}{\large{\faicon{group}}} & \hyperlink{lines130}{\huge{\faBook}} \\
\hline
   \large{31} & & & \hyperlink{blank31}{\large{\faicon{group}}} & \hyperlink{lines131}{\huge{\faBook}} \\
\hline
   \large{32} & & & \hyperlink{blank32}{\large{\faicon{group}}} & \hyperlink{lines132}{\huge{\faBook}} \\
\hline
   \large{33} & & & \hyperlink{blank33}{\large{\faicon{group}}} & \hyperlink{lines133}{\huge{\faBook}} \\
\hline
   \large{34} & & & \hyperlink{blank34}{\large{\faicon{group}}} & \hyperlink{lines134}{\huge{\faBook}} \\
\hline
   \large{35} & & & \hyperlink{blank35}{\large{\faicon{group}}} & \hyperlink{lines135}{\huge{\faBook}} \\
\hline
   \large{36} & & & \hyperlink{blank36}{\large{\faicon{group}}} & \hyperlink{lines136}{\huge{\faBook}} \\
\hline

\end{tabular}

\newpage

% Content Generation as before
\foreach \n in {1,...,18}{
    % Blank Page
    \newpage
    \thispagestyle{empty}
    \hypertarget{blank\n}{}
    \mbox{} % Ensures the page is truly seen as blank but used
    \foreach \t in {1,...,5}{
          \begin{tikzpicture}[remember picture, overlay]
            \node[anchor=west, fill=black, text=white, rounded corners=2mm, minimum height=10mm, minimum width=10mm, font=\Large] at ([xshift=-3mm, yshift=-\t*15mm]current page.north west) {\hyperlink{lines\t\n}{\faicon{paper-plane-o}}};
          \end{tikzpicture}
        }
        \begin{tikzpicture}[remember picture, overlay]
            \node[anchor=west, fill=black, text=white, rounded corners=2mm, minimum height=10mm, minimum width=10mm, font=\Large] at ([xshift=-3mm, yshift=-10*15mm]current page.north west) {\hyperlink{index1}{\faicon{list}}};
          \end{tikzpicture}
    
    % Five pages with lines
    \foreach \m in {1,...,5}{
        \newpage
        \thispagestyle{empty}
        \hypertarget{lines\m\n}{}
        % Draw lines
        \begin{tikzpicture}[overlay,remember picture]
            \pgfmathsetmacro{\lineSpace}{0.8}
            \pgfmathtruncatemacro{\numberOfLines}{\paperheight/\lineSpace/1cm}
            \foreach \i in {1,...,\numberOfLines}{
                \pgfmathsetmacro{\yPos}{\i*\lineSpace}
                \draw [line width=0.2mm] ($(current page.south west)+(0,\yPos)$) -- ($(current page.south east)+(0,\yPos)$);
            }
        \end{tikzpicture}
        \begin{tikzpicture}[remember picture, overlay]
            \node[anchor=west, fill=black, text=white, rounded corners=2mm, minimum height=10mm, minimum width=10mm, font=\Large] at ([xshift=-1mm, yshift=-1*15mm]current page.north west) {\hyperlink{blank\n}{\faicon{group}}};
            \node[anchor=west, fill=black, text=white, rounded corners=2mm, minimum height=10mm, minimum width=10mm, font=\Large] at ([xshift=-1mm, yshift=-2*15mm]current page.north west) {\hyperlink{todolist}{\faicon{check-square}}};
          \end{tikzpicture}
        \mbox{} % To ensure the page is not empty for TikZ drawing
        \begin{tikzpicture}[remember picture, overlay]
            \node[anchor=west, fill=black, text=white, rounded corners=2mm, minimum height=10mm, minimum width=10mm, font=\Large] at ([xshift=-3mm, yshift=-10*15mm]current page.north west) {\hyperlink{index1}{\faicon{list}}};
          \end{tikzpicture}
    }
}

\foreach \n in {19,...,36}{
    % Blank Page
    \newpage
    \thispagestyle{empty}
    \hypertarget{blank\n}{}
    \mbox{} % Ensures the page is truly seen as blank but used
    \foreach \t in {1,...,5}{
          \begin{tikzpicture}[remember picture, overlay]
            \node[anchor=west, fill=black, text=white, rounded corners=2mm, minimum height=10mm, minimum width=10mm, font=\Large] at ([xshift=-3mm, yshift=-\t*15mm]current page.north west) {\hyperlink{lines\t\n}{\faicon{paper-plane-o}}};
          \end{tikzpicture}
        }
        \begin{tikzpicture}[remember picture, overlay]
            \node[anchor=west, fill=black, text=white, rounded corners=2mm, minimum height=10mm, minimum width=10mm, font=\Large] at ([xshift=-1mm, yshift=-10*15mm]current page.north west) {\hyperlink{index2}{\faicon{list}}};
          \end{tikzpicture}
    
    % Five pages with lines
    \foreach \m in {1,...,5}{
        \newpage
        \thispagestyle{empty}
        \hypertarget{lines\m\n}{}
        % Draw lines
        \begin{tikzpicture}[overlay,remember picture]
            \pgfmathsetmacro{\lineSpace}{0.8}
            \pgfmathtruncatemacro{\numberOfLines}{\paperheight/\lineSpace/1cm}
            \foreach \i in {1,...,\numberOfLines}{
                \pgfmathsetmacro{\yPos}{\i*\lineSpace}
                \draw [line width=0.2mm] ($(current page.south west)+(0,\yPos)$) -- ($(current page.south east)+(0,\yPos)$);
            }
        \end{tikzpicture}
        \begin{tikzpicture}[remember picture, overlay]
            \node[anchor=west, fill=black, text=white, rounded corners=2mm, minimum height=10mm, minimum width=10mm, font=\Large] at ([xshift=-1mm, yshift=-1*15mm]current page.north west) {\hyperlink{blank\n}{\faicon{group}}};
            \node[anchor=west, fill=black, text=white, rounded corners=2mm, minimum height=10mm, minimum width=10mm, font=\Large] at ([xshift=-1mm, yshift=-2*15mm]current page.north west) {\hyperlink{todolist}{\faicon{check-square}}};
          \end{tikzpicture}
        \mbox{} % To ensure the page is not empty for TikZ drawing
        \begin{tikzpicture}[remember picture, overlay]
            \node[anchor=west, fill=black, text=white, rounded corners=2mm, minimum height=10mm, minimum width=10mm, font=\Large] at ([xshift=-1mm, yshift=-10*15mm]current page.north west) {\hyperlink{index2}{\faicon{list}}};
          \end{tikzpicture}
    }
}

\end{document}
